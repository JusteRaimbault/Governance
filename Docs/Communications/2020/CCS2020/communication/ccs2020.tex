\input{header_slides.tex}

\begin{document}

\title
[A co-evolution ABM for systems of cities]{A co-evolution agent-based model for systems of cities and transportation networks integrating top-down governance through game theory}
\author[Raimbault]{J.~Raimbault$^{1,2,3}$\\\medskip
$^{\ast}$\texttt{juste.raimbault@polytechnique.edu}
}

\institute[UCL]{$^{1}$Center for Advanced Spatial Analysis, University College London\\
$^{2}$UPS CNRS 3611 Complex Systems Institute Paris\\
$^{3}$UMR CNRS 8504 G{\'e}ographie-cit{\'e}s
}




\date[08/12/2020]{CCS 2020\\
\\
December 8th, 2020
}

\frame{\maketitle}


% The evolutionary theory for systems of cities at the macroscopic scale proposed by [1] suggests the existence of co-evolutionary dynamics in the trajectories of cities and their environment. In particular, transportation infrastructure connecting cities can in some cases co-evolve with them [2]. Understanding such processes is crucial for sustainable planning at large scales. The issue of the interplay between bottom-up emergence of urban dynamics and top-down planning of infrastructures is in that context relevant to study. We propose in this contribution a model of co-evolution for cities and transportation networks, with a focus on how transportation networks evolve. More particularly, we extend the model of [3] by introducing top-down governance agents which decide on investments in transportation links. The model simulates population trajectories of cities and network link speed trajectories, with two main modules: (i) spatial interaction modeling to determine growth rates of cities, and (ii) governance modeling for network evolution. Using a game-theoretic approach, macroscopic agents (such as governments or planning authorities) arbitrate stochastically between national and international investments, following a payoff-matrix considering optimal accessibility gains and collaboration costs, with probabilities obtained under the assumption of mixed strategies in a Nash equilibrium. Network investments are used to increase effective link speed by fixed increments. The model is applied to synthetic systems of cities, in a stylized configuration of two neighbor countries of comparable size. We systematically study model behavior with the OpenMOLE platform for model exploration and validation [4]. First exploration results suggest a strong qualitative influence of propensity to collaborate on trajectories of the full system, and that intermediate levels of international investments may be more optimal in terms of accessibility gains at fixed costs. In comparison to null model behavior obtained running the base model from [3], the introduction of top-down governance decisions also changes considerably model behavior. We also show that initial spatial conditions such as urban hierarchy significantly influence model outcomes [5]. This work illustrates how co-evolution models at this scale can be refined, opening research possibilities towards more complex or multi-scale models.
% [1] Pumain, D. (2018). An evolutionary theory of urban systems. In International and transnational perspectives on urban systems (pp. 3-18). Springer, Singapore.
%[2] Raimbault, J. (2020). Modeling the co-evolution of cities and networks. In Handbook of Cities and Networks. Rozenblat C., Niel Z., eds. In press
%[3] Raimbault, J. (2020). Hierarchy and co-evolution processes in urban systems. arXiv preprint arXiv:2001.11989.
%[4] Reuillon, R., Leclaire, M., & Rey-Coyrehourcq, S. (2013). OpenMOLE, a workflow engine specifically tailored for the distributed exploration of simulation models. Future Generation Computer Systems, 29(8), 1981-1990.
%[5] Raimbault, J., Cottineau, C., Le Texier, M., Le Nechet, F., & Reuillon, R. (2019). Space Matters: Extending Sensitivity Analysis to Initial Spatial Conditions in Geosimulation Models. Journal of Artificial Societies and Social Simulation, 22(4).



\section{Introduction}


\sframe{Urban evolutionary theory}{

\cite{pumain2018evolutionary}

}


\sframe{Co-evolution and governance processes}{

\cite{Xie2011}
\cite{xie2011governance}

}

\sframe{Multinational transport investments}{


\cite{melo2013productivity} meta-analysis: positive effects of transport investments

\cite{tsamboulas1984multinational} difficulty of multinational investments

\cite{tsamboulas2007tool} framework for prioritization, using multi attribute theory

\cite{yii2018transportation} one belt one road - positive effect of transport investments

\cite{proost2014selected} many trans European projects fail in cost-benefit analysis


}

\sframe{Game theory and transportation}{

\cite{adler2010high} competition HSR/airplane

\cite{medda2007game} public private partnerships

\cite{roumboutsos2008game} public transport integration

}


\sframe{Proposed approach}{


\cite{raimbault2020modeling}

}



\section{Model}


\sframe{Model rationale}{

extend the model of \cite{raimbault2020hierarchy}

}


\sframe{Cities populations}{

%The model operates at the macroscopic scale: basic units are cities, distributed in space and described by their population $P_i$. A spatial interaction model is used to estimate flows between pairs of cities as

\[
\varphi_{ij} = \left( P_i P_j \right)^{\gamma} \cdot \exp\left(- \frac{d_{ij}}{d_0}\right)
\]

% From these flows is computed a growth rate for cities, assuming that it is for one city proportional to the sum of flows to all other cities, such that

\[
\frac{P_i (t+ \Delta t) - P_i(t)}{\Delta t} \propto P_i(t)^{\gamma} \cdot \sum_j P_j(t)^{\gamma} \cdot \exp\left(- \frac{d_{ij}}{d_0}\right)
\]

% WITH FIXED EFFECT COUNTRY MULTIPLIER

% Flows between cities are then distributed into the network. We assume no congestion and that the shortest path is taken.


}

\sframe{Network}{

%The set of deciders - which are the level of respective countries - evaluate the relevance of different infrastructure development options. More precisely, they have in a simplified setting the choice between an international collaboration and a strictly national infrastructure development. Let $Z^{\ast}_i$ be the optimal infrastructure in terms of accessibility gains for the country $i$ only, and $Z^{\ast}_C$ the optimal international infrastructure.
%Making the assumption that deciders aim at optimizing the accessibility gain, 

% - distinguish between national and international flows
% - link increase function: flow quantile, improve only if above (no decrease)



}


\sframe{Payoff matrix}{

\begin{tabular}{ |c|c|c| } 
 \hline
 0 $|$ 1  & C & NC \\ \hline
 C & $U_i = \Delta X_i(Z^{\ast}_C) - I - \frac{J}{2}$
   & $\begin{cases}U_0 = \Delta X_0(Z^{\ast}_0)-I \\U_1 = \Delta X_1(Z^{\ast}_1)-I - \frac{J}{2}\end{cases}$ \\ \hline
 NC & $\begin{cases}U_0 =  \Delta X_0(Z^{\ast}_0)-I - \frac{J}{2}\\U_1 =\Delta X_1(Z^{\ast}_1)-I\end{cases}$
   & $U_i = \Delta X_i(Z^{\ast}_i) - I$ \\
 \hline
\end{tabular}

}


\sframe{Mixed Nash equilibrium probabilities}{

\[
p_{1-i} = - \frac{U_i(C,NC) - U_i(NC,NC)}{\left(U_i(C,C) - U_i(NC,C)\right) - \left(U_i(C,NC) - U_i(NC,NC)\right)}
\]

\begin{equation}
p_i = \frac{J}{\Delta X_{1 - i}{Z^{\star}_{C}} - \Delta X_{1 - i}{Z^{\star}_{1 - i}}}
\end{equation}

% from lutecia paper
% It also forces feasibility conditions on $J$ and accessibility gains to keep a probability. These are
%\begin{itemize}
%	\item $ J \leq \Delta X_{1 - i}(Z^{\star}_{C}) - \Delta X_{1 - i}(Z^{\star}_{1 - i})$, what can be interpreted as a cost-benefits condition for cooperation, e.g. that the gain induced by the common infrastructure must be larger than the collaboration cost;
%	\item $\Delta X_{1 - i}(Z^{\star}_{C}) \leq \Delta X_{1 - i}(Z^{\star}_{1 - i})$, e.g. that the gain induced by the common infrastructure must be positive.
%\end{itemize}

}



\section{Results}

\sframe{Implementation and experiments}{

\footnotesize

\textbf{Implementation}

\begin{itemize}
	%\item Model implemented in NetLogo (good compromise interactivity / ergonomy), with fast data structures (matrix/table extensions)
	%\item Integrated seamlessly into OpenMOLE \cite{reuillon2013openmole} for model exploration \url{https://openmole.org/}
\end{itemize}

\begin{center}
\includegraphics[width=0.1\linewidth]{figures/iconOM.png}
\includegraphics[width=0.4\linewidth]{figures/openmole.png}
\end{center}

\medskip

%\textbf{Experiments}
%
%\begin{itemize}
%	\item 
%\end{itemize}

}




\sframe{Role of urban hierarchy}{

\cite{raimbault2019space}

}


\section{Discussion}

\sframe{Discussion}{

% Discuss alternative flow assignment procedures. You can in particular have a look at the transportation literature

% Discuss the structure of this model, of canonical LUTI, and of a sample of more advanced LUTI models could a much more detailed representation of underlying capture intrinsic crucial process which would be missed by our approach? In other words, what is the purpose of LUTI models? More generally, discuss the relation between the degree of validation of a model and its potential operationalization.



\justify

\textbf{Developments}

\medskip

$\rightarrow$ 

\medskip

$\rightarrow$ 

\bigskip
\bigskip


\textbf{Applications}

\medskip

$\rightarrow$ 


}








\sframe{Conclusion}{


$\rightarrow$

\bigskip

$\rightarrow$


\bigskip
\bigskip

\textbf{Open repositories}

\url{https://github.com/JusteRaimbault/CoevolGov}

\url{https://github.com/JusteRaimbault/Governance}


\bigskip

\textbf{Data at} \texttt{}

}



%%%%%%%%%%%%%%%%%%%%%
\begin{frame}[allowframebreaks]
\frametitle{References}
\bibliographystyle{apalike}
\bibliography{biblio}
\end{frame}
%%%%%%%%%%%%%%%%%%%%%%%%%%%%



%\sframe{Reserve slides}{
%
%\Huge
%
%\centering
%
%Reserve slides
%
%}



\end{document}

