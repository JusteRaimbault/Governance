
\documentclass{beamer}
\usetheme{ucl}

\usepackage[utf8]{inputenc}


%%% Increase the height of the banner: the argument is a scale factor >=1.0
%\setbeamertemplate{banner}[ucl][10.0]

%%% Change the colour of the main banner
%%% The background should be one of the UCL colours (except pink or white):
%%%   black,darkpurple,darkred,darkblue,darkgreen,darkbrown,richred,midred,
%%%   navyblue,midgreen,darkgrey,orange,brightblue,brightgreen,lightgrey,
%%%   lightpurple,yellow,lightblue,lightgreen,stone
\setbeamercolor{banner}{bg=darkpurple}
%\setbeamercolor{banner}{bg=yellow,fg=black}

%%% Add a stripe behind the banner
%\setbeamercolor{banner stripe}{bg=darkpurple,fg=black}

%%% The main structural elements
\setbeamercolor{structure}{fg=black}

%%% Author/Title/Date and slide number in the footline
\setbeamertemplate{footline}[author title date]

%%% Puts the section/subsection in the headline
% \setbeamertemplate{headline}[section]

%%% Puts a navigation bar on top of the banner
%%% For this to work correctly, the each \section command needs to be
%%% followed by a \subsection. Requires one extra compile.
% \setbeamertemplate{headline}[miniframes]
%%% Accepts an optional argument determining the width
% \setbeamertemplate{headline}[miniframes][0.3\paperwidth]


%%% Puts the frame title in the banner
%%% Won't work correctly with the above headline templates
%\useoutertheme{ucltitlebanner}
%%% Similar to above, but smaller (and puts subtitle on same line as title)
\useoutertheme[small]{ucltitlebanner}

%%% Gives block elements (theorems, examples) a border
% \useinnertheme{blockborder}
%%% Sets the body of block elements to be clear
% \setbeamercolor{block body}{bg=white,fg=black}

%%% Include CSML logo on title slide
%\titlegraphic{\includegraphics[width=0.16\paperwidth]{csml_logo}}

%%% Include CSML logo in bottom right corner of all slides
%\logo{\includegraphics[width=0.12\paperwidth]{csml_logo}}

%%% Set a background colour
% \setbeamercolor{background canvas}{bg=lightgrey}

%%% Set a background image
%%% Some sample images are available from the UCL image store:
%%%   https://www.imagestore.ucl.ac.uk/home/start
% \setbeamertemplate{background canvas}{%
%   \includegraphics[width=\paperwidth]{imagename}}



%%%%%% Some other settings that can make things look nicer
%%% Set a smaller indent for description environment
\setbeamersize{description width=2em}
%%% Remove nav symbols (and shift any logo down to corner)
\setbeamertemplate{navigation symbols}{\vspace{-2ex}}








\DeclareMathOperator{\Cov}{Cov}
\DeclareMathOperator{\Var}{Var}
\DeclareMathOperator{\E}{\mathbb{E}}
\DeclareMathOperator{\Proba}{\mathbb{P}}

\newcommand{\Covb}[2]{\ensuremath{\Cov\!\left[#1,#2\right]}}
\newcommand{\Eb}[1]{\ensuremath{\E\!\left[#1\right]}}
\newcommand{\Pb}[1]{\ensuremath{\Proba\!\left[#1\right]}}
\newcommand{\Varb}[1]{\ensuremath{\Var\!\left[#1\right]}}

% norm
\newcommand{\norm}[1]{\| #1 \|}

\newcommand{\indep}{\rotatebox[origin=c]{90}{$\models$}}





\usepackage{mathptmx,amsmath,amssymb,graphicx,bibentry,bbm,ragged2e}
\usepackage[english]{babel}

\makeatletter

\newcommand{\noun}[1]{\textsc{#1}}
\newcommand{\jitem}[1]{\item \begin{justify} #1 \end{justify} \vfill{}}
\newcommand{\sframe}[2]{\frame{\frametitle{#1} #2}}

\newenvironment{centercolumns}{\begin{columns}[c]}{\end{columns}}
%\newenvironment{jitem}{\begin{justify}\begin{itemize}}{\end{itemize}\end{justify}}



%\usetheme{Warsaw}
%\setbeamertemplate{footline}[text line]{}
%\setbeamertemplate{headline}{}
%\setbeamercolor{structure}{fg=purple!50!blue, bg=purple!50!blue}

%\setbeamersize{text margin left=15pt,text margin right=15pt}

%\setbeamercovered{transparent}


\@ifundefined{showcaptionsetup}{}{%
 \PassOptionsToPackage{caption=false}{subfig}}
\usepackage{subfig}

\usepackage[utf8]{inputenc}
\usepackage[T1]{fontenc}

\usepackage{multirow}


\makeatother

\def \draft {1}

\usepackage{xparse}
\usepackage{ifthen}
\DeclareDocumentCommand{\comment}{m o o o o}
{\ifthenelse{\draft=1}{
    \textcolor{red}{\textbf{C : }#1}
    \IfValueT{#2}{\textcolor{blue}{\textbf{A1 : }#2}}
    \IfValueT{#3}{\textcolor{ForestGreen}{\textbf{A2 : }#3}}
    \IfValueT{#4}{\textcolor{red!50!blue}{\textbf{A3 : }#4}}
    \IfValueT{#5}{\textcolor{Aquamarine}{\textbf{A4 : }#5}}
 }{}
}
\newcommand{\todo}[1]{
\ifthenelse{\draft=1}{\textcolor{red!50!blue}{\textbf{TODO : \textit{#1}}}}{}
}




\begin{document}

\title
[A co-evolution ABM for systems of cities]{A co-evolution agent-based model for systems of cities and transportation networks integrating top-down governance through game theory}
\author[Raimbault]{J.~Raimbault$^{1,2,3}$\\\medskip
$^{\ast}$\texttt{juste.raimbault@polytechnique.edu}
}

\institute[UCL]{$^{1}$Center for Advanced Spatial Analysis, University College London\\
$^{2}$UPS CNRS 3611 Complex Systems Institute Paris\\
$^{3}$UMR CNRS 8504 G{\'e}ographie-cit{\'e}s
}




\date[08/12/2020]{CCS 2020\\
\\
December 8th, 2020
}

\frame{\maketitle}


% The evolutionary theory for systems of cities at the macroscopic scale proposed by [1] suggests the existence of co-evolutionary dynamics in the trajectories of cities and their environment. In particular, transportation infrastructure connecting cities can in some cases co-evolve with them [2]. Understanding such processes is crucial for sustainable planning at large scales. The issue of the interplay between bottom-up emergence of urban dynamics and top-down planning of infrastructures is in that context relevant to study. We propose in this contribution a model of co-evolution for cities and transportation networks, with a focus on how transportation networks evolve. More particularly, we extend the model of [3] by introducing top-down governance agents which decide on investments in transportation links. The model simulates population trajectories of cities and network link speed trajectories, with two main modules: (i) spatial interaction modeling to determine growth rates of cities, and (ii) governance modeling for network evolution. Using a game-theoretic approach, macroscopic agents (such as governments or planning authorities) arbitrate stochastically between national and international investments, following a payoff-matrix considering optimal accessibility gains and collaboration costs, with probabilities obtained under the assumption of mixed strategies in a Nash equilibrium. Network investments are used to increase effective link speed by fixed increments. The model is applied to synthetic systems of cities, in a stylized configuration of two neighbor countries of comparable size. We systematically study model behavior with the OpenMOLE platform for model exploration and validation [4]. First exploration results suggest a strong qualitative influence of propensity to collaborate on trajectories of the full system, and that intermediate levels of international investments may be more optimal in terms of accessibility gains at fixed costs. In comparison to null model behavior obtained running the base model from [3], the introduction of top-down governance decisions also changes considerably model behavior. We also show that initial spatial conditions such as urban hierarchy significantly influence model outcomes [5]. This work illustrates how co-evolution models at this scale can be refined, opening research possibilities towards more complex or multi-scale models.
% [1] Pumain, D. (2018). An evolutionary theory of urban systems. In International and transnational perspectives on urban systems (pp. 3-18). Springer, Singapore.
%[2] Raimbault, J. (2020). Modeling the co-evolution of cities and networks. In Handbook of Cities and Networks. Rozenblat C., Niel Z., eds. In press
%[3] Raimbault, J. (2020). Hierarchy and co-evolution processes in urban systems. arXiv preprint arXiv:2001.11989.
%[4] Reuillon, R., Leclaire, M., & Rey-Coyrehourcq, S. (2013). OpenMOLE, a workflow engine specifically tailored for the distributed exploration of simulation models. Future Generation Computer Systems, 29(8), 1981-1990.
%[5] Raimbault, J., Cottineau, C., Le Texier, M., Le Nechet, F., & Reuillon, R. (2019). Space Matters: Extending Sensitivity Analysis to Initial Spatial Conditions in Geosimulation Models. Journal of Artificial Societies and Social Simulation, 22(4).



\section{Introduction}



\section{Discussion}

\sframe{Discussion}{

\justify

\textbf{Developments}

\medskip

$\rightarrow$ 

\medskip

$\rightarrow$ 

\bigskip
\bigskip


\textbf{Applications}

\medskip

$\rightarrow$ 


}








\sframe{Conclusion}{


$\rightarrow$

\bigskip

$\rightarrow$


\bigskip
\bigskip

\textbf{Open repositories}

\url{https://github.com/JusteRaimbault/CoevolGov}

\url{https://github.com/JusteRaimbault/Governance}


\bigskip

\textbf{Data at} \texttt{}

}



%%%%%%%%%%%%%%%%%%%%%
\begin{frame}[allowframebreaks]
\frametitle{References}
\bibliographystyle{apalike}
\bibliography{biblio}
\end{frame}
%%%%%%%%%%%%%%%%%%%%%%%%%%%%



%\sframe{Reserve slides}{
%
%\Huge
%
%\centering
%
%Reserve slides
%
%}



\end{document}

